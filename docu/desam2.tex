%% =======================================================
%% Project:
%%
%% Description:
%%
%% File: desam2.tex
%% Path: c:/Users/scheidan/Dropbox/Eawag/DeSaM2/docu/
%%
%% July 19, 2013 -- Andreas Scheidegger
%%
%% andreas.scheidegger@eawag.ch
%% =======================================================


\documentclass[11pt, a4paper, parskip=half*, captions=tableheading]{scrartcl}

% --- use packages
\usepackage[T1]{fontenc}                % right coding
\usepackage{lmodern}                    % otherwise \usepackage[T1]{fontenc} uses bitmap fonts
\usepackage[USenglish]{babel}		% 
\usepackage{graphicx} 			% to include figures
\usepackage{amssymb}			% mathe...
\usepackage{amsmath}
\usepackage{color}
\usepackage{microtype}			% for better type setting
\usepackage{tabularx}			% for tables
\usepackage{float}                      % enables to force a float enviroment to be excactly here with  [H]
\usepackage[hang,small,bf]{caption}     % options for captions
\usepackage{natbib}                     % author�year citations
%\usepackage{textcomp}			% symbol for celsius
%\usepackage{pdfpages}			% to include external pdfs with \includepdf[pages={-}]{...pdf}
%\usepackage{attachfile}                % attach files into pdf document
%\usepackage{setspace}			% set linespacing: \begin{onehalfspacing} \begin{singlespacing}
%\usepackage{lineno}                    % for line numbering

\usepackage[ansinew]{inputenc}	        % to type 'Umlaute; directly. On Windows.
%\usepackage[applemac]{inputenc} 	%          "                  On OS X.

\definecolor{darkblue}{rgb}{0,0,0.5}
\usepackage[colorlinks=true,
            linkcolor=darkblue,
            urlcolor=darkblue,
            citecolor=darkblue]{hyperref}% pdf-links

% some other fonts
\usepackage{cmbright}                   % better readable on screen
% \usepackage{palatino}
% \usepackage{ccfonts} 

\usepackage{ellipsis}			% correct \dots

% --- options 

\setcounter{secnumdepth}{3}             % how deep should the sections be numerated
\setcounter{tocdepth}{1}                % how deep should the sections be
                                        % listed in the 
% define a command for comments
\newcommand{\comment}[1] {
 \marginpar{
   \begin{flushleft}
     {\footnotesize
       $\leftarrow$ \textit{comment:} \newline #1
     }
   \end{flushleft}
 }
}


% =========================================================
% Title

\title{DeSaM -- Julia implementation} 
\date{\today}
\author{\href{mailto: scheidegger.a@gmail.com}{Andreas Scheidegger}}


\begin{document}
\maketitle                      

\tableofcontents

% ============================================
\section{Introductions}

The Decentralized Sanitation Model (DeSaM) was originaly developed and
implemented in \textsl{R} by Thomas Hug. 

However, the \textsl{R} implementation is very slow and software
design is somewhat flawed so that bug-fixing and implementation of new
features became almost impossible. Therefore, DeSaM was reimplemented
from scratch in \href{http://julialang.org/}{Julia}, a ``high-level,
high-performance dynamic programming language for technical
computing''.

The aim of the reimplementation is not to reproduce all features of
the \textsl{R} version but rather provide a well designed basic structure that can be
extended easily by the user.

% ============================================
\section{Design}

A DeSaM simulation consist only of interacting \texttt{Tank}
objects. The tanks are arranged in a tree like structure. A tank can
be filled either directly by a source (e.g.\ a household) or by
collecting from upstream tanks.


% ---------------------------------
\subsection{Tanks}

Each \texttt{Tank} object (see Figure~\ref{fig:tank}) is defined by
its maximal volume, the initial costs, a source function, a collection
function, and, if existing, a vector of upstream \texttt{Tank}
objects. A tank has tree state variables that are updated
over time: the tank volume, the overflow volume, and the costs.

\begin{figure}[h]
  \centering
  \includegraphics[width=0.5\textwidth]{figures/tank.pdf}
  \caption{Structure of a \texttt{Tank} object.}
  \label{fig:tank}
\end{figure}


The type definition shows all fields\footnote{A ``field'' is an element
  of a composit type, e.g. \texttt{V\_max}. The expected type of a
  field is indicated after the ``\texttt{::}''~operator.} of a \texttt{Tank}
object in detail:
{\small
\begin{verbatim}
type Tank

    ## tank properties
    V_max::Float64

    ## State variables
    V::Float64
    V_overflow::Float64
    costs::Float64
    time::Int

    ## function for sources, takes 'time' as argument
    ## and returns (volume, costs)
    source::Function

    ## function for collection, takes 'upstream tanks' and 'time' as arguments
    ## and returns (volume, costs)
    collection::Function

    ## a vector of elements of type 'tank'
    has_upstream_tanks::Bool
    upstream_tanks::Vector{Tank}

end
\end{verbatim}
}


Three functions are available to create \texttt{Tank} objects:
{\small
\begin{verbatim}
Tank(V_max::Real, source::Function, costs::Real)
Tank(V_max::Real, upstream_tanks::Vector, collection::Function, costs::Real)
Tank(V_max::Real, upstream_tanks::Vector, collection::Function, source::Function, costs::Real)
\end{verbatim}
} The first option is used when a tank has no upstream tanks and
threrefore no collection neither. The second function is typically used for
 collection tanks. The last function defines a tank that serves as
collection tank and has a direct source at the same time.

Initial costs of the tank \emph{and}, if existing, of the
collection are defined by \texttt{costs}.

The behavior of a tank is largely influenced by its source and
collection functions.  In most cases the user will provide these
functions. However, some basic functions are implemented that can be
modified or used directly. 


\subsubsection{source function}
The source function can be any function that takes as argument
\texttt{time} and returns a tuple with the daily volume that goes in
the tank and the daily costs.

A very simple source function is: 
{\small 
\begin{verbatim}
function simple_source(time)
    ## produced volume of all household members
    V = rand()*10 # U(0, 10)
    ## costs
    costs = 0.1*V
    return(V, costs)
end
\end{verbatim}
}

\subsubsection{collection function}
The collection function can be any function that takes as argument a
vector of \texttt{Tank} objects and \texttt{time}. It must return a
tuple with the daily collected volume and the daily costs.
\textbf{The function must also change the volume of each parent tank
  that is emptied!}

A minimalistic example: 
{\small 
\begin{verbatim}
function simple_collection(tanks::Vector{Tank}, time)
    V_coll_max = 200.0
    n_tanks_max = 10
    n_tanks = size(tanks,1)

    ## tanks are emptied the same order
    V_coll = 0.0
    for i in 1:min(n_tanks, n_tanks_max)
        V_tank_out = min(tanks[i].V, V_coll_max - V_coll)
        tanks[i].V -= V_tank_out   # !!! change volume of parent tank !!!
        V_coll += V_tank_out     
    end

    ## costs
    costs = 10.0 + 0.2*V_coll

    return(V_coll, costs)
end
\end{verbatim}
}

\subsection{Simulation}

After defining a network of tanks the simulation can be performed. This
is done by updating the \emph{last} tank via \texttt{update()}.
\texttt{update()} makes use of the recursive structure and updates all
upstream tanks  automatically.

A minimal example of a simulation:{\small
\begin{verbatim}
t_sim_max = 365
for t in 1:t_sim_max
    ## update final tank
    update(tank_final)             
end
\end{verbatim}
}

Typically, on is interested how some state variables evolve over
time. A more realistic example where every day the total overflow
volume of the household tanks is stored (the function \texttt{push!(A, e)} adds
element \texttt{e} to the end of array \texttt{A}): {\small
\begin{verbatim}
## define an empty vector to store results
Volumes_overflow_households = Float64[]   # type must be given

t_sim_max = 365
for t in 1:t_sim_max
    ## update final tank
    update(tank_final)   

    ## compute sum of all overflows
    V_overflow = sum(get_field_of_parent_tanks(tank_final, 1, :V_overflow)) 
    ## save every time step
    push!(Volumes_overflow_households, V_overflow) 
          
end
\end{verbatim}
}


% ============================================
\section{Example}

As an extended example it is illustrated how the system  in
Figure~\ref{fig:tree} is implemented.

\begin{figure}[h]
  \centering
  \includegraphics[width=0.8\textwidth]{figures/tank_tree.pdf}
  \caption{Tank structure of the example with four different types of tanks.}
  \label{fig:tree}
\end{figure}

\subsection{Tank definition}

The first step is the definition of all tanks.
{\small
\begin{verbatim}
## -------------------------------------------------------
## 1) Define tanks
## -------------------------------------------------------

## --- tanks A ---

## Volume:         10 liters
## upstream_tanks: -
## collection:     -
## source:         household_source; 10 people, 1.5 liter/day/person median
## initial costs:  100.00 

tanks_A = [Tank(10, def_household_source(10, 1.5), 100.00) for i=1:6]
## creats a vector of 6 similar tanks


## --- tank B ---

## Volume:         50 liters
## upstream_tanks: tanks_A
## collection:     random collection;  max 5 tanks or max 20L, every 2nd day
## source:         household_source; 10 people, 1.5 liter/day/person median
## initial costs:  250.00 (tank) + 100.00 (collection)

tank_B = Tank(50,
              tanks_A,
              def_random_collection(5, 20, 2),
              def_household_source(10, 1.5),
              250.00+100.00)
show(tank_B)


## --- tanks C ---

## Volume:         20 liters
## upstream_tanks: -
## collection:     -
## source:         household_source; 15 people, 1.5 liter/day/person median
## initial costs:  150.00 each

tanks_C = [Tank(20, def_household_source(15, 1.5), 150.00) for i=1:4]
## creats a vector of 4 similar tanks


## --- tank D ---

## Volume:         150 liters,
## upstream_tanks: [tank_B, tanks_C]
## collection:     random collection;  max 5 tanks or max 200L, every 3nd day
## source:         -
## initial costs:  500.00 (tank) + 200.00 (collection)

tank_D = Tank(150,
              [tank_B, tanks_C],
              def_random_collection(5, 200, 3),
              500.00 + 200.00)
show(tank_D)
\end{verbatim}
}

Here the functions \texttt{def\_random\_collection()} and
\texttt{def\_household\_source()} are used to define conveniently
different source ans collection function, i.e.\ are these functions
that return a function. \texttt{show()} prints a summary of 
properties of tank.

Note the use of the
\href{http://docs.julialang.org/en/release-0.1/manual/arrays/#comprehensions}{comprehension}
syntax to produce arrays of similar tanks.


\subsection{simulation}
{\small
\begin{verbatim}
## -------------------------------------------------------
## 2) run simulation
## -------------------------------------------------------

println("setup costs: ", total_costs(tank_D))

## define empty vectors to save results
costs_tank_D = Float64[]
V_tanks_A = Float64[]

t_sim_max = 10*365                      # simulate 10 years
for t in 1:t_sim_max

    ## update last tank only
    update(tank_D)          

    ## -- three ways to obtain sum of all overflows of tanks A
    Vol = sum(get_field_of_tanks(tanks_A, :V)) # directly from tanks_A
    Vol = sum(get_field_of_upstream_tanks(tank_B, 0, :V)) # tanks_A are the
                                                          #   upstream tanks of tank B
    Vol = sum(get_field_of_upstream_tanks(tank_D, 1, :V)) # tanks_A are the
                                                          #   upstream tanks of tank D at level 1

    ## write results in a vector
    push!(V_tanks_A, Vol)
    push!(costs_tank_D, tank_D.costs) # costs, only of tank D (no costs of parent tanks)

end


## print results
println("Total costs after 10 years: ", total_costs(tank_D))
println("Average costs of tank D: ", mean(costs_tank_D))
println("Average volume off all tanks A: ", mean(V_tanks_A))


## --- write results to file ---
writecsv("output/output.csv", [V_tanks_A costs_tank_D])
\end{verbatim}
}

% ============================================
\section{Limitations of the design}

The chosen design enables a very elegant and fast implementation in
Julia. So is the core functionality implemented with less than 150 lines
of code. Most missing features should be easy to add. However, there
are intrinsic limitation of the design (not sure how far they apply to the
\textsl{R} version, too):

\begin{itemize}
\item The collections cannot be organized ``globally'' because they
  depend only on the tanks to empty and the time.
\item Multiple substances could be implemented, however, different
  collecting routes for different substances are not possible
\item Only tree like systems can be simulated. Also distribution is not
  possible, i.e. to transport volume form one tank to several
  downstream tanks.
\end{itemize}

%\citet{le_gat_2009} said that \citep[e.g.][]{le_gat_2009}.

% \begin{figure}[t]
%   \centering
%   \includegraphics[width=11cm]{figures/fig.pdf}
%   \caption{here you see}
%   \label{fig:fig}
% \end{figure}


% % =======================================================
% \section{References}

% % References with bibTeX database:
% \bibliographystyle{elsarticle-harv}
% \bibliography{myReferences}


\end{document}