%% =======================================================
%% Project:
%%
%% Description:
%%
%% File: desam2.tex
%% Path: c:/Users/scheidan/Dropbox/Eawag/DeSaM2/docu/
%%
%% July 19, 2013 -- Andreas Scheidegger
%%
%% andreas.scheidegger@eawag.ch
%% =======================================================


\documentclass[11pt, a4paper, parskip=half*, captions=tableheading]{scrartcl}

% --- use packages
\usepackage[T1]{fontenc}                % right coding
\usepackage{lmodern}                    % otherwise \usepackage[T1]{fontenc} uses bitmap fonts
\usepackage[USenglish]{babel}		% 
\usepackage{graphicx} 			% to include figures
\usepackage{amssymb}			% mathe...
\usepackage{amsmath}
\usepackage{color}
\usepackage{microtype}			% for better type setting
\usepackage{tabularx}			% for tables
\usepackage{float}                      % enables to force a float enviroment to be excactly here with  [H]
\usepackage[hang,small,bf]{caption}     % options for captions
\usepackage{natbib}                     % author�year citations
%\usepackage{textcomp}			% symbol for celsius
%\usepackage{pdfpages}			% to include external pdfs with \includepdf[pages={-}]{...pdf}
%\usepackage{attachfile}                % attach files into pdf document
%\usepackage{setspace}			% set linespacing: \begin{onehalfspacing} \begin{singlespacing}
%\usepackage{lineno}                    % for line numbering

\usepackage[ansinew]{inputenc}	        % to type 'Umlaute; directly. On Windows.
%\usepackage[applemac]{inputenc} 	%          "                  On OS X.

\definecolor{darkblue}{rgb}{0,0,0.5}
\usepackage[colorlinks=true,
            linkcolor=darkblue,
            urlcolor=darkblue,
            citecolor=darkblue]{hyperref}% pdf-links

% some other fonts
\usepackage{cmbright}                   % better readable on screen
% \usepackage{palatino}
% \usepackage{ccfonts} 

\usepackage{ellipsis}			% correct \dots

% --- options 

\setcounter{secnumdepth}{3}             % how deep should the sections be numerated
\setcounter{tocdepth}{1}                % how deep should the sections be
                                        % listed in the 
% define a command for comments
\newcommand{\comment}[1] {
 \marginpar{
   \begin{flushleft}
     {\footnotesize
       $\leftarrow$ \textit{comment:} \newline #1
     }
   \end{flushleft}
 }
}


% =========================================================
% Title

\title{DeSaM -- Julia implementation} 
\date{\today}
\author{\href{mailto: scheidegger.a@gmail.com}{Andreas Scheidegger}}


\begin{document}
\maketitle                      

\tableofcontents

% ============================================
\section{Introductions}

The Decentralized Sanitation Model (DeSaM) was originaly developed and
implemented in \textsl{R} by Thomas Hug. 

However, the \textsl{R} implementation is very slow and the design of
the software is flawed so that bug-fixing and implementation of new
features became almost impossible. Therefore, DeSaM was reimplemented
from scratch in \href{http://julialang.org/}{Julia}, a ``high-level,
high-performance dynamic programming language for technical
computing''.

The aim of the reimplementation is not to reproduce all features of
the \textsl{R} version but rather provide a well designed basic structure that can be
extended easily by the user.

% ============================================
\section{Design}

A DeSaM simulation consist on a number of interacting \texttt{Tank}
objects that are arranged in a tree like structure.

% ---------------------------------
\subsection{Tank}

Each \texttt{Tank} object is defined by its maximal volume, a source
function, a collection function, and, if existing, a vector of parent
\texttt{Tank} objects. Direct upstream tanks that are emptied in a tank
are referred as ``parent tanks''.

The exact type definition:
{\small
\begin{verbatim}
type Tank

    ## tank properties
    V_max::Float64

    ## State variables
    V::Float64
    V_overflow::Float64
    costs::Float64
    time::Int

    ## function for sources, takes 'time' as argument
    ## and returns (volume, costs)
    source::Function

    ## function for collection, takes 'parent tanks' and 'time' as arguments
    ## and returns (volume, costs)
    collection::Function

    ## a vector of elements of type 'tank'
    has_parents::Bool
    parents::Vector{Tank}

end
\end{verbatim}
}

With exception of \texttt{costs} all internal state variables are
automatically initialized, so the user does not have to provide
initial values. However, all state variables can be read during
the simulation.

\subsubsection{source function}
The source function can be any function that takes as argument
\texttt{time} and returns a tuple with the daily volume that goes in
the tank and the daily costs.

A very minimalist source function could be: 
{\small 
\begin{verbatim}
function simple_source(time)
    ## produced volume of all household members
    V = rand()*10 # U(0, 10)
    ## costs
    costs = 0.1*V
    return(V, costs)
end
\end{verbatim}
}

\subsubsection{collection function}
The collection function can be any function that takes as argument a
vector of \texttt{Tank} objects and \texttt{time}. It must return a
tuple with the daily collected volume and the daily costs.

The function must also change the volume of the parent tank when it is emptied!

A minimalist example: 
{\small 
\begin{verbatim}
function simple_collection(tanks::Vector{Tank}, time)
    V_coll_max = 200.0
    n_tanks_max = 10

    n_tanks = size(tanks,1)

    ## tanks are emptied the same order
    V_coll = 0.0
    for i in 1:min(n_tanks, n_tanks_max)
        V_tank_out = min(tanks[i].V, V_coll_max - V_coll)
        tanks[i].V -= V_tank_out   # !!! change volume of parent tank !!!
        V_coll += V_tank_out     
    end

    ## costs
    costs = 10.0 + 0.2*V_coll
    return(V_coll, costs)
end
\end{verbatim}
}

\subsection{Simulation}

After defining a network of tanks a simulation can be performed. This
is done by updating the \emph{last tank} via \texttt{update()}.
\texttt{update()} makes use of the recursive structure and updates all upstream tanks  automatically.

A minimal example of a simulation:{\small
\begin{verbatim}
t_sim_max = 365
for t in 1:t_sim_max
    ## update final tank
    update(tank_final)             
end
\end{verbatim}
}

Typically, on is interested in how some state variables evolve over
time. So a more realistic example is (\texttt{push!(A, e)} adds element \texttt{e} to the end of array \texttt{A}): 
{\small
\begin{verbatim}
## define empty vector to store results
Volumes_overflow_households = Float64[]

t_sim_max = 365
for t in 1:t_sim_max
    ## update final tank
    update(tank_final)   

    ## compute sum of all overflows
    V_overflow = sum(get_field_of_parent_tanks(tank_final, 1, :V_overflow)) 
    ## save every time step
    push!(Volumes_overflow_households, V_overflow) 
          
end
\end{verbatim}
}



% ============================================
\section{Limitations of the design}

The chosen design enables a very elegant and fast implementation in
Julia. The core functionality is implemented with less than 150 lines
of code. More missing features should be easy to add. However, there
are some intrinsic limitation of the design (not sure if apply to the
\textsl{R} version, too):

\begin{itemize}
\item The collections cannot be organized ``globally'' because they
  depend only on the tanks to empty and the time.
\item Multiple substances could be implemented, however, different
  collecting routes for different substances are not possible
\item Only tree like systems can be simulated. Also distribution is not
  possible, i.e. to transport volume to several child tanks.
\end{itemize}

%\citet{le_gat_2009} said that \citep[e.g.][]{le_gat_2009}.

% \begin{figure}[t]
%   \centering
%   \includegraphics[width=11cm]{figures/fig.pdf}
%   \caption{here you see}
%   \label{fig:fig}
% \end{figure}


% % =======================================================
% \section{References}

% % References with bibTeX database:
% \bibliographystyle{elsarticle-harv}
% \bibliography{myReferences}


\end{document}